% FONTS UND SPRACHE
\usepackage{fontspec}
\usepackage{polyglossia}
\setdefaultlanguage[spelling=new,babelshorthands=true]{german}

% MATHEMATIK UND NATURWISSENSCHAFT
\usepackage{mathtools}					% loads amsmath, e.g. pmatrix*
\usepackage{unicode-math}				% e.g. mathbb
\usepackage{siunitx}		            % SI-Unit package


% GRAFIK UND BILD
\usepackage{graphicx}					% bitmap graphics
\usepackage{tikz}						% vector graphics
\usetikzlibrary{calc}
\usepackage{pgfplots}					% plotting
\pgfplotsset{compat=1.18}
\usepgfplotslibrary{groupplots}


% LAYOUT UND DESIGN
\usepackage{scrlayer-scrpage}			% koma script package
\usepackage{tabularx}					% full width tables
\usepackage{makecell}					% controll tablecell size
\usepackage[most]{tcolorbox}	        % fancy boxes
\usepackage{subcaption}
\usepackage{pageslts}


% TEXTERWEITERUNGEN UND SONSTIGES
\usepackage[plainpages=false,pdfpagelabels]{hyperref}	% urls
\usepackage{bookmark} 					%
\usepackage{cancel}						% cancel chars
\usepackage{comment}					% comments
\usepackage[normalem]{ulem}		        % underline
\usepackage{environ}					% better environment
\usepackage{qrcode}						% generate qrcodes
\usepackage{xcolor}						% colors
\usepackage{emoji}						% insert emoji



%*********************
% INITIALISIERUNGEN
%*********************

% siunitx
\sisetup{
	output-decimal-marker = {.},
	per-mode = symbol,
	separate-uncertainty = false,
	add-decimal-zero = true,
	exponent-product = \cdot,
	round-mode=none
}

% hyperref
\hypersetup{pdfencoding=auto}
\hypersetup{
  pdftitle={\mytitle},
  pdfauthor={\myauthor},
  pdfsubject={\mysubject},
  pdfkeywords={\mykeywords}
}

% set * to \cdot in math mode
\mathcode`\*="8000
\begingroup
  \catcode`\*=\active
  \gdef*{\cdot}
\endgroup



%*******************
% LAYOUT
%*******************

% Nummerierung der Sections ohne Punkt (z. B. "1.1 Einleitung")
\renewcommand{\thesection}{\thechapter.\arabic{section}}
\renewcommand{\thesubsection}{\thesection.\arabic{subsection}}

% Keine Nummern in Kopfzeilen
\renewcommand*{\sectionmarkformat}{}

% Kopfzeilen-Schrift normal (nicht kursiv)
\setkomafont{pageheadfoot}{\normalfont}

% Abstand Kopf → Text
\headsep=1cm

% Scrlayer-Scrpage Kopf- und Fusszeilen aktivieren
\clearpairofpagestyles
\pagestyle{scrheadings}
\automark[section]{section}

% Kopfzeilen links/mitte/rechts (für gerade/ungerade Seiten identisch)
\ihead{\myauthor}
\chead{\headmark}
\ohead{\mydate}

% Fusszeile mit Seitenzahl z. B. 5 / 24
\pagenumbering{arabic}
\cfoot{\hfill \thepage\ / \lastpages{arabic}{1}}



%************************
%* LAYOUT KONFIGURATION *
%************************

% Nummerierungen
\renewcommand{\thefigure}{\arabic{figure}}
\renewcommand{\thetable}{\arabic{table}}
\renewcommand{\theequation}{\arabic{equation}}

% Beschriftungen (deutsch)
\addto\captionsgerman{%
  \renewcommand{\figurename}{Abb.}%
  \renewcommand{\tablename}{Tab.}%
}

% Kurzreferenzen
\newcommand{\fref}[1]{Abb.~\ref{#1}}
\newcommand{\tref}[1]{Tab.~\ref{#1}}

% Setze Emoji Schriftart
\setemojifont{TwemojiMozilla}



%**************************
% DEFINITIONEN UND MAKROS *
%**************************

\newtcolorbox{greenbox}{
    enhanced,
    boxrule=0pt,frame hidden,
    borderline west={4pt}{0pt}{green!75!black},
    colback=green!10!white,
    sharp corners
}

\newtcolorbox{redbox}{
    enhanced,
    boxrule=0pt,frame hidden,
    borderline west={4pt}{0pt}{red!75!black},
    colback=red!10!white,
    sharp corners
}



\newcounter{exercise}
\setcounter{exercise}{1}
\newcommand{\exercise}[2][pencil]{
    \subsection*{\makebox(0,0)[r]{\huge\emoji{#1}\hspace{1em}}Aufgabe \theexercise\  -- #2}
    \addtocounter{exercise}{1}
}

\newcounter{experiment}
\setcounter{experiment}{1}
\newcommand{\experiment}[2][screwdriver]{
    \subsection*{\makebox(0,0)[r]{\rotatebox[origin=c]{270}{\huge\emoji{#1}}\hspace{1em}}Experiment \theexperiment\  -- #2}
    \addtocounter{experiment}{1}
}


\newcommand{\marker}{\fontspec{Marker Felt}\fontsize{5mm}{8mm}\selectfont}

\newcommand{\markertable}{\fontspec{Marker Felt}\fontsize{5mm}{12mm}\selectfont}


\newif\ifshowanswers
\showanswerstrue % oder \showanswersfalse

\newcommand{\answer}[1]{\ifshowanswers #1 \fi}
\newcommand{\noanswer}[1]{\ifshowanswers \else #1 \fi}
